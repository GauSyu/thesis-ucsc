% LTeX: enabled=false
%%-----------------------------------------
%% Macros for notations
%%-----------------------------------------
%
% Math Alphabeta
% Delimiters
% Relations
% MathOperators
% Standard notations
% Others
%
%%-----------------------------------------
%% Creat a bunch of commands
%%-----------------------------------------
\ExplSyntaxOn
% #1 = command to use
% #2 = optional prefix
% #3 = list
\NewDocumentCommand{\createbunch}{ m O{} m }
	{
		\clist_map_inline:nn { #3 } { \cs_new_protected:cpn { #2 ##1 } { #1 { ##1 } } }
	}
\cs_new:Nn \cs_createbunch:nnN 
	{
		\clist_map_inline:Nn #3 { \cs_new_protected:cpn { #2 ##1 } { #1 { ##1 } } }
	}
%
%%-----------------------------------------
%% Math Alphabeta
%%-----------------------------------------
%%% A list of upper case letters
\clist_const:Nn \g_Alphabeta_clist 
	{A,B,C,D,E,F,G,H,I,J,K,L,M,N,O,P,Q,R,S,T,U,V,W,X,Y,Z} 
%%% A list of lower case letters 
\clist_const:Nn \g_alphabeta_clist 
	{a,b,c,d,e,f,g,h,i,j,k,l,m,n,o,p,q,r,s,t,u,v,w,x,y,z} 
%%% \mathbf{*} -> \bf*
\cs_createbunch:nnN {\mathbf} {bf} \g_Alphabeta_clist 
\cs_createbunch:nnN {\mathbf} {bf} \g_alphabeta_clist
%%% \mathbb{*} -> \bb*
\cs_createbunch:nnN {\mathbb} {bb} \g_Alphabeta_clist 
\cs_createbunch:nnN {\mathbb} {bb} \g_alphabeta_clist
%%% \mathcal{*} -> \cal*
\cs_createbunch:nnN {\mathcal} {cal} \g_Alphabeta_clist 
%%% \mathscr{*} -> \scr*
\cs_createbunch:nnN {\mathscr} {scr} \g_Alphabeta_clist 
\cs_createbunch:nnN {\mathscr} {scr} \g_alphabeta_clist 
%%% \mathfrak{*} -> \frak*
\cs_createbunch:nnN {\mathfrak} {frak} \g_Alphabeta_clist 
\cs_createbunch:nnN {\mathfrak} {frak} \g_alphabeta_clist 
%%% \mathsf{*} -> \sf*
\cs_createbunch:nnN {\mathsf} {sf} \g_Alphabeta_clist 
\cs_createbunch:nnN {\mathsf} {sf} \g_alphabeta_clist 
\ExplSyntaxOff
\providecommand\bm[1]{\boldsymbol{#1}} % Bold symbols
%
%%-----------------------------------------
%% Delimiters
%%-----------------------------------------
%%% one-variable
\DeclarePairedDelimiterX\abs[1]\lvert\rvert
	{ \ifblank{#1}{\:\cdot\:}{#1} } % |...|
\DeclarePairedDelimiterX\norm[1]\lVert\rVert
	{ \ifblank{#1}{\:\cdot\:}{#1} } % ||...||
\DeclarePairedDelimiterX\arga[1]\langle\rangle
	{ \ifblank{#1}{\:\cdot\:}{#1} } % <...>
\DeclarePairedDelimiterX\argp[1]\lparen\rparen
	{ \ifblank{#1}{\:\cdot\:}{#1} } % (...)
\DeclarePairedDelimiterX\argo[1]\lbrack\rbrack
	{ \ifblank{#1}{\:\cdot\:}{#1} } % [...]
\DeclarePairedDelimiterX\argdb[1]\dlb\drb
	{ \ifblank{#1}{\:\cdot\:}{#1} } % [[]...]]
\DeclarePairedDelimiterX\argm[1]\lbrace\rbrace
	{ \ifblank{#1}{\:\cdot\:}{#1} } % {...}
\DeclarePairedDelimiterX\ceil[1]\lceil\rceil
	{ \ifblank{#1}{\:\cdot\:}{#1} } % ceiling
\DeclarePairedDelimiterX\floor[1]\lfloor\rfloor
	{ \ifblank{#1}{\:\cdot\:}{#1} } % floor
\DeclarePairedDelimiterX\bra[1]\langle\vert
	{ \ifblank{#1}{\:\cdot\:}{#1} } % <...|
\DeclarePairedDelimiterX\ket[1]\vert\rangle
	{ \ifblank{#1}{\:\cdot\:}{#1} } % |...>
\NewDocumentCommand\normord{m}{\mathopen{:}#1\mathclose{:}} % :...:
%%% bi-variable
\DeclarePairedDelimiterX\braket[2]\langle\rangle
	{
		\ifblank{#1}{\:\cdot\:}{#1}
		\,\delimsize\vert\,\mathopen{}
		\ifblank{#2}{\:\cdot\:}{#2}
	} % <..|..>
\DeclarePairedDelimiterX\pairing[2]\langle\rangle
	{ \ifblank{#1}{\:\cdot\:}{#1}, \ifblank{#2}{\:\cdot\:}{#2} } % <..,..>
\DeclarePairedDelimiterX\inner[2]\lparen\rparen
	{ \ifblank{#1}{\:\cdot\:}{#1}, \ifblank{#2}{\:\cdot\:}{#2} } % (..,..)
%%% Sets
\providecommand\given{} % emp delimiter
\NewDocumentCommand \SetSymbol {o}
  { \nonscript\:#1\vert\allowbreak\nonscript\:\mathopen{} }  % delimiter in set
\DeclarePairedDelimiterX\Set[1]\{\}
  { \renewcommand\given{\SetSymbol[\delimsize]}#1 } % Set
\DeclarePairedDelimiterX\GSet[1]\langle\rangle
  { \renewcommand\given{\SetSymbol[\delimsize]}#1 } % Generator
%
%%-----------------------------------------
%% Relations
%%-----------------------------------------
%%% sub/supsets
\let\oldsubset\subset
\let\oldsupset\supset
\RenewDocumentCommand \subset {} {\subseteq}
\NewDocumentCommand \stsubset {} {\oldsubset}
\RenewDocumentCommand \supset {} {\supseteq}
\NewDocumentCommand \stsupset {} {\oldsupset}
%%% le/ge
\RenewDocumentCommand \le {} { \leqslant }
\RenewDocumentCommand \ge {} { \geqslant }
% 
%%-----------------------------------------
%% MathOperators
%%-----------------------------------------
%%% Functions
% \NewDocumentCommand \fun { O{p} m e{^_} O{} }
%   {%
%     \operatorname{#2}%
%     \IfValueT{#3}{\sp{#3}}%
%     \IfValueT{#4}{\sb{#4}}%
%     \IfBlankTF{#5}{}{\csname arg#1\endcsname*{#5}}%
%   }
\NewDocumentCommand \fun { m e{^_} O{} }
  {%
    \operatorname{#1}%
    \IfValueT{#2}{\sp{#2}}%
    \IfValueT{#3}{\sb{#3}}%
    \ifblank{#4}{}{\mleft(#4\mright)}%
  }
%%% other fonts
\NewDocumentCommand \mfrak { m } {\fun{\mathfrak{#1}}}
\NewDocumentCommand \mcal { m } {\fun{\mathcal{#1}}}
\NewDocumentCommand \mscr { m } {\fun{\mathscr{#1}}}
\NewDocumentCommand \msf { m } {\fun{\mathsf{#1}}}
\NewDocumentCommand \mbf { m } {\fun{\mathbf{#1}}}
\NewDocumentCommand \mbb { m } {\fun{\mathbb{#1}}}
\NewDocumentCommand \grp { m } {\fun{\mathsf{#1}}}
\NewDocumentCommand \cat { m } {\fun{\mathbf{#1}}}
\NewDocumentCommand \sheaf { m } {\fun{\mathscr{#1}}}
% 